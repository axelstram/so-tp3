\section{Introducci\'on}

La tem\'atica de trabajo se centra en el paradigma de c\'omputo Map-reduce, que da soporte a la computaci\'on paralela sobre grandes colecciones de datos.
\vspace{2mm}


El contenido de este trabajo se divide en dos secciones: la primera parte de implementaci\'on y experimentaci\'on, con la implementaci\'on de varias funciones MapReduce en $Javascript$, y $querys$ bajo el sistema de base de datos No-SQL $MongoDB$. La segunda parte se basa en el an\'alisis del documento cient\'ifico \textbf{Job Scheduling for Multi-User MapReduce Clusters: }, de la universidad  de California de Berkeley. 

\vspace{2mm}

